\section{Quasi-conformal Maps and Measures of Non-degenerateness}

For a map \(f: \mathbb R^2 \to \mathbb R^2\), we may measure how non-degenerate the map is by using the 
constant of quasi-conformality. In a sense, this measures how far the map is from "degenerating" into a
one-dimensional image; note that the constant of quasi-conformality is scale invariant as such a measure
should be.

However, keep in mind that we can still have branch points, e.g. the conformal (and complex) map \(g(z) = z^2\)
has a degnerate branch point at \(z = 0\). Despite this, the image will be missing degenerate behavior such as 
one-dimensional boundaries (unless it is the image of the boundary of definition).
