\section{The Harnack Inequality as a Quantitative Maximum Principle}

Caffarelli \cite{Caffarelli1988} says that we should interpret the Harnack Inequality as a quantitative version of
the maximum principle. Here we describe a simple version of this interpretation, especially for the classic Harnack
Inequality.

Let's consider a simple version of the Harnack Inequality for non-negative harmonic functions. That is, let \(u\geq0\)
on \(\Omega\) and \(\delta u = 0\) on a bounded domain \(\Omega\). In this case, the Harnack Inequality is actually a quantitative version
of the \textit{Minimum Principle}.

To see so, let us further assume that \(u(x_0) = 0\) for some point on \(x_0 \in \partial \Omega\) (so \(u\) is in some sense on the
extreme version of non-negative harmonic functions on \(\Omega\)). Recall that the Minimum Principle then says that \(u > 0\) in   
\(\Omega\) if \(u\) is non-constant (i.e. \(u \neq 0\) on \(\Omega\)).

However, the Harnack Inequality tells us more. To see this, recall a simple version of the Harnack Inequality, 
\(\sup\limits_{\Omega'} u \leq C \inf\limits_{\Omega'} \) for any domain \(\Omega' \subset\subset \Omega\) and constant \(C\) depending
on \(\Omega'\) and \(\Omega\). So, while the Minimum Principle tells us that \(u > 0\) on \(\Omega'\), it doesn't give us any information
on how large this gap is independent of \(u\). However, the Harnack Inequality tells us that the gap between \(u\) and \(0\) on \(\Omega'\)
can be uniformly bounded below depending only on the relationship between \(\Omega'\) and \(\Omega\); that is, we can't get \(u\) arbitralily
close to \(0\) on \(\Omega'\) by varying the values of \(u\) on \(\partial \Omega\).
