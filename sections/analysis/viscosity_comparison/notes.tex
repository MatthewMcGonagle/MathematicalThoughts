\section{Viscosity Solutions and Comparison Principles}

\subsection*{Remark on Sobolev Solutions}

When we design weak solutions of a differential equation, we ask that the solutions retain some of
the properties of the classical solutions. For example, Sobolev functions provide weak solutions
to divergence form elliptic equations where we retain certain integral identities that exist for classical
solutions. For Sobolev functions, the function itself and its derivatives are connected enough to still
allow the application of integration by parts, when applicable. For example, a classical harmonic function
\(u\) and a Sobolev harmonic function \(u\) both satisfy
\begin{equation}
0 = \int\limits_\Omega \langle \nabla u, \nabla \phi \rangle dV = -\int_\Omega u \Delta \phi dV, 
\end{equation} 
for any \(\phi \in C^2_0(\Omega)\).

In particular, we can use the left hand equality applied to \(\phi = u \eta^2\) for some \(\eta\in C^1_0(\Omega)\).
This can be used to establish interior integral estimates for \(\|\nabla u\|\); we see that we get
\begin{equation}
\int\limits_\Omega \|\nabla u\|^2 \eta^2 dV = - \int\limits_\Omega 2 u \eta \langle \nabla u, \nabla \eta \rangle dV. 
\end{equation}
Using Cauchy-Schwarz, we then have that 
\begin{equation}
\int\limits_\Omega \|\nabla u\|^2 \eta^2 dV \leq 4 \int\limits_\Omega u^2 \|\nabla \eta\|^2 dV.
\end{equation}
With a standard choice of \(eta\), this quickly gives us interior gradient estimates
\begin{equation}
\int\limits_{B_\rho(0)} \|\nabla u\|^2 dV \leq \frac{4}{\rho^2} \int\limits_{B_{2\rho}(0)} u^2 dV. 
\end{equation}

\subsection*{Viscosity Solutions}

For elliptic equations, one way to interpret \textit{viscosity solutions} is that they allow us to
retain a comparison principle. In fact, the comparison principle is baked into the definition; one can
also use this viewpoint to keep track of the inequalities and their directions in the definitions of 
viscosity solutions, sub-solutions, and super-solutions.

So it is no surprise that techniques using viscosity solutions lean heavily on comparison principles.
