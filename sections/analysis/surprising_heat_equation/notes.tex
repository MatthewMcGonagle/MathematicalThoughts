\section{The Suprising Existence and Regularity of the Heat Equation}

Here we briefly comment as to why the existence and regularity of the heat equation can be considered suprising.
Thinking of the heat equation as an evolution in time, one may be lead to the intuition that the solution
exists simply by "evolving" forward at each time \(t\); this is the intuition of short time existence for 
first order ODE. However, here we show that the impact of the spatial derivative \(u_{xx}\) on regularity
gives us a competing intuition for the solution not existing even under nice conditions. So not only is it
suprising that the heat equation has smooth solutions under nice conditions, but that it has 
solutions at all.

Let us constider the simple case of the heat equation 
\begin{equation}
\begin{cases}
u_t = u_{xx} & -1 < x < 1 \text{ and } t > 0, \\
u(-1, t) = -2 & t > 0, \\
u(1, t) = -2 & t > 0, \\
u(x, 0) = x^2 (|x| - 3) & 0 \leq x \leq 1.
\end{cases}
\end{equation}
Note that the boundary conditions are continuous. Furthermore, the first order boundardy condition 
\(u_{xx}(\pm 1, 0) = 0 = u_t(\pm 1, t)\) is satisfied as well. Note also that \(\phi(x) = u(x, 0)\) is
in \(C^{2, 1}\), in particular we have that \(\phi_{xx} = 6|x| - 6\) is Lipschitz. 

What is suprising about this case is the heat equations regularity and existence in the interior at \(x = 0\)
and \(t > 0\). Consider the first order approximation 
\begin{align}
u(x, 0 + h) & \approx u(x, 0) + u_t(x, 0)h, \\
    & = u(x, 0) + u_xx(x, 0)h, \\
    & = \phi(x) + 6(|x| - 1)h.
\end{align}
While \(\phi(x) \in C^{2, 1}([-1, 1])\), the last summand (when freezing \(h\)) \(6(|x| - 1)h\) is only in 
\(C^{0, 1}([-1, 1])\). Therefore, it seems one iteration of the evolution forward in time destroys
two derivatives of the spatial differentiability. It is therefore suprising that the solution \(u(x, t)\) should
be defined or even \(C^2\) for any \(t > 0\). 

If instead we had started with an initial condition \(u(x, 0) = \psi(x)\) with \(\psi \in C^k([-1, 1]\) for any
\(k\), we would still run into the same problem. Look at \(m\) first order approximations where \(2m > k\), and
we run into the same problem (actually we may run into the problem slightly before the \(m\)th iteration).
